\documentclass[a4paper,12pt]{article}

\usepackage{phypreamble}

\solutiontrue

\begin{document}

\begin{gather}
    \Psi(\vec{r}, t) \in \CC \\
    \left| \Psi \right|^2 = p(\vec{r}, t)\ \text{(плотность вероятности)}, \int_{\RR^3} \left| \Psi \right|^2 \dd V = 1 \label{prop:psi-1} \\
    \Psi(\vec{r}, t)\ \text{определена с точностью до}\ \exp(i \xi) \\
    \Psi\ \text{ограничена и непрерывна} \\
    \Psi = \alpha_1 \Psi_1 + \alpha_2 \Psi_2\ \text{--- линейность}
\end{gather}

\begin{problem}{}
\[
    \Psi(x) =
    \begin{cases}
        A \sin \frac{\pi x}{a}, & 0 \leq x \leq a \\
        0, & \text{иначе}
    \end{cases}
\]

Определить \(A\).

\begin{solution}
Воспользуемся свойством \ref{prop:psi-1}:
\[A^2 \int_0^a \sin^2 \frac{\pi x}{a} \dd x = 1.\]
\[\frac{A^2}{2} \int_0^a 1 - \cos \frac{2\pi x}{a} \dd x = \frac{A^2}{a} = 1 \implies A = \sqrt{\frac{2}{a}}.\]
\end{solution}

\end{problem}

\begin{lemma}[Гауссов интеграл]
\begin{equation}\label{eq:gauss-int}
I = \int_{\RR} \exp\left(-t^2\right) \dd t = \sqrt\pi.
\end{equation}
\begin{proof}
\[I^2 = \int_{\RR^2} \exp\left(-x^2 - y^2\right) \dd x \dd y \by{\mathrm{ПСК}} \int_0^{2\pi} \int_0^{+\infty} \exp\left(-\rho^2\right) \rho \dd \rho \dd \varphi = 2 \pi \cdot \frac12 \int_0^{+\infty} \exp\left(-\rho^2\right) \dd \rho^2 = \pi. \]
\end{proof}
\end{lemma}

\begin{problem}{}
\[\Psi(x) = A \exp\left(-\frac{x^2}{2 x_0^2}\right).\]
Определить \(A\).

\begin{solution}
Воспользуемся свойством \ref{prop:psi-1}:
\[A^2 \int_{\RR} \exp\left(-\frac{x^2}{x_0^2}\right) \dd x = \left[ t = \frac{x}{x_0} \right] = A^2 x_0 \int_{\RR} \exp\left(-t^2\right) \dd t = 1.\]
По \eqref{eq:gauss-int} \(I = \int_{\RR} \exp\left(-t^2\right) \dd t = \sqrt\pi \).
Тогда \(A = \frac{1}{\sqrt{x_0 \sqrt{\pi}}}\).
\end{solution}
\end{problem}


\begin{problem}{}
\[\Psi(x) = A x \exp\left(-\frac{x^2}{2 x_0^2}\right).\]
Определить \(A\).

\begin{solution}
Воспользуемся свойством \ref{prop:psi-1}:
\[A^2 \int_{\RR} x^2 \exp\left(-\frac{x^2}{x_0^2}\right) \dd x = \left[ t = \frac{x}{x_0} \right] = A^2 x_0^2 \int_{\RR} t^2 \exp\left(-t^2\right) \dd t = 1.\]
Введём параметр \(\alpha\):
\[
\begin{split}
    &\left.\int_{\RR} t^2 \exp\left(-\alpha t^2\right) \dd t\right|_{\alpha = 1}
    = \left[-\frac{\partial}{\partial \alpha}\left(\int_{\RR} \exp\left(-\alpha t^2\right) \dd t \right)\right]_{\alpha = 1} \\
    &= \left[-\frac{\partial}{\partial \alpha}\left(\frac1{\sqrt{\alpha}} \int_{\RR} \exp\left(-\left(\sqrt{\alpha} t\right)^2\right) \dd \left(\sqrt{\alpha} t\right)\right)\right]_{\alpha = 1}
    = \left[-\frac{\partial}{\partial \alpha}\left(\frac1{\sqrt{\alpha}} I \right)\right]_{\alpha = 1}
    = \left[-\frac{\partial}{\partial \alpha}\left(\sqrt{\frac{\pi}{\alpha}}\right)\right]_{\alpha = 1} \\
    &= \left.\frac12 \sqrt\pi \alpha^{-3/2}\right|_{\alpha = 1} = \frac{\sqrt\pi}{2}.
\end{split}
\]

Тогда \(A = \frac{\sqrt{2}}{x_0 \sqrt[4]{\pi}}\).
\end{solution}
\end{problem}

\section{Коммутаторы}

Коммутатором для операторов \(\hat{A}\) и \(\hat{B}\) является
\[ \left[\hat{A}, \hat{B}\right] = \hat{A}\hat{B} - \hat{B}\hat{A}.\]
Операторы \(\hat{A}\) и \(\hat{B}\) коммутируют, если \(\left[\hat{A}, \hat{B}\right] = 0\).

Некоторые операторы:
\begin{align*}
    \hat{r} &= \vec{r} && \text{оператор координаты} \\
    \hat{\vec{p}} &= -i \hbar \nabla && \text{оператор импульса} \\
    \hat{T} &= \frac{\hat{\vec{p}}^2}{2m} = -\frac{\hbar \Laplace}{2 m} && \text{оператор кинетической энергии} \\
    \hat{U} &= U && \text{оператор потенциальной энергии} \\
    \HHam &= \hat{T} + \hat{U} && \text{оператор полной энергии}
\end{align*}

\begin{problem}{}
Вычслить \(\left[\hat{x}, \hat{p}_x \right]\).

\begin{solution}
\[
\begin{split}
    & \left[\hat{x}, \hat{p}_x \right] \Psi
    = \hat{x} \hat{p}_x \Psi - \hat{p}_x \hat{x} \Psi
    = \hat{x} \left(-i\hbar \frac{\partial \Psi}{\partial x}\right) - \left(- i \hbar \frac{\partial \left(x \Psi\right)}{\partial x}\right)
    = -i \hbar \left(x \frac{\partial \psi}{\partial x} - \frac{\partial \left(x \Psi\right)}{\partial x} \right) \\
    &= -i \hbar \left(x \frac{\partial \psi}{\partial x} - \Psi - x \frac{\partial Psi}{\partial x}\right)
    = i \hbar \Psi.
\end{split}
\]
Следовательно, \(\left[\hat{x}, \hat{p}_x \right] = i \hbar\).
\end{solution}
\end{problem}

\begin{problem}{}
\begin{align*}
    \left[\hat{x}, \hat{p}_y \right] &= 0 \\
    \left[\hat{x}, \hat{y} \right] &= 0 \\
    \left[\hat{p}_i, \hat{p}_j \right] &= 0 \\
    \left[\hat{x}_i, \hat{x}_j \right] &= 0 \\
    \left[\hat{x}_i, \hat{p}_j \right] &= i\hbar \delta_{ij}
\end{align*}
\end{problem}

\begin{problem}{}
Вычислить \(\left[\hat{\vec{r}}, \hat{\vec{p}}^2 \right]\).

\begin{solution}
Некоторые свойства операторов:
\begin{gather}
    \left[\alpha \hat{A} + \beta \hat{B}, \hat{C}\right] = \alpha \left[\hat{A}, \hat{C}\right] + \beta \left[\hat{B}, \hat{C}\right]  \label{prop:commutator-linear}\\
    \left[\hat{A}, \hat{B} \hat{C}\right] = \left[\hat{A}, \hat{B}\right] \hat{C} + \hat{B} \left[\hat{A}, \hat{C}\right] \label{prop:abc-abc-bac}.
\end{gather}

Тогда
\[
\begin{split}
    & \left[\op{\vec{r}}, \hat{\op{p}}^2 \right]
    = \left[\sum_k \hat{x}_k \vec{e}_k, \sum_l \hat{p}_l^2\right]
    = \sum_{k, l} \vec{e}_k \left[\hat{x}_k, \hat{p}_l \hat{p}_l \right]
    = \sum_{k, l} \vec{e}_k \left(\left[\hat{x}_k, \hat{p}_l\right] \hat{p}_l + \hat{p}_l \left[\hat{x}_k, \hat{p}_l\right]\right) \\
    &= \sum_{k, l} \vec{e}_k \cdot 2 i \hbar \delta_{k l} \hat{p}_l
    = 2 i \hbar \sum_i \hat{p}_i \vec{e}_i = 2 i \hbar \hat{\vec{p}}.
\end{split}
\]

Доказательство свойства \ref{prop:commutator-linear}:
\[
  \left[\alpha \op A + \beta \op B, \op C \right]
  = \left(\alpha \op A + \beta \op B\right) \op C - \op C \left(\alpha \op A + \beta \op B\right)
  = \alpha \left(\op A \op C - \op C \op A \right) + \beta \left(\op B \op C - \op C \op B\right)
  = \alpha \left[\op A, \op C\right] + \beta \left[\op B, \op C\right].
\]

Доказательство свойства \ref{prop:abc-abc-bac}:
\[
\begin{split}
  & \comm{\op A}{\op B \op C}
  = \op A \op B \op C - \op B \op C \op A
  = \op A \op B \op C - \op B \op A \op C + \op B \op A \op C - \op B \op C \op A \\
  &= \left(\op A \op B - \op B \op A \right) \op C + \op B \left(\op A \op C - \op C \op A \right)
  = \comm{\op A}{\op B} \op C + \op B \comm{\op A}{\op C}.
\end{split}
\]
\end{solution}
\end{problem}

\section{Эрмитов оператор}

\begin{definition}[Сопряженной оператор]
\({\op F}^+\) --- сопряженный оператор к \(\op F\), если
\begin{equation}\label{prop:her-adjoint}
\forall \Phi, \Psi:\ \left({\op F}^+ \Psi, \Phi \right) = \left(\Psi, \op F \Phi \right)
\end{equation}

В данном случае скалярное произведение определено так:
\[
  \left(f, g\right) = \int_{\RR} f(x) g(x)^* \dd x.
\]
\end{definition}

\begin{definition}[Эрмитов оператор]
\(\op F\) --- эрмитов оператор, если \({\op F}^+ = \op F\).
\end{definition}

\begin{definition}[Среднее значение оператора]
\[
\avgOp{F} = \int_{\RR} \Psi^* \op F \Psi \dd x = \left(\op F \Psi, \Psi\right).
\]
\end{definition}

\begin{remark}
Пусть у оператора \(\op F\) есть собственные векторы \(\Psi_n\) и соответствующие собственные числа \(f_n\).
Тогда произвольную волновую функциную можно разложить по собственным векторам:
\[\Psi = \sum_n C_n \Psi_n,\]
где \(C_n = \left(\Psi, \Psi_n\right)\).

Среднее значение можно определить как:
\[\avgOp{f} = \sum_n P_n f_n,\]
где \(P_n\) --- вероятность.

Можно считать, что \(P_n = \left|C_n\right|^2\), тогда
\[
\begin{split}
  & \avgOp{f}
  = \sum_n \left|C_n\right|^2 f_n
  = \sum_n C_n^* C_n f_n
  = \sum_n C_n \left(\Psi_n, \Psi\right) f_n
  = \sum_n C_n \left(f_n \Psi_n, \Psi\right)
  = \sum_n C_n \left(\op F \Psi_n, \Psi\right) \\
  &= \left(\sum_n C_n \op F \Psi_n, \Psi\right)
  = \left(\op F \sum_n C_n \Psi_n, \Psi\right)
  = \left(\op F \Psi, \Psi\right).
\end{split}
\]

\end{remark}

\begin{theorem}
\[
\avgOp{F} \in \RR \iff \op{F}^+ = \op F.
\]
\begin{proof}
\[
  \avgOp{F}^* = \left(\Psi, \op F \Psi \right) \by{\eqref{prop:her-adjoint}} \left(\op F \Psi, \Psi\right) = \avgOp{F}.
\]
\[
  \avgOp{F}^* = \int_{\RR} \Psi \op{F}^* \Psi^* \dd x \by{\eqref{prop:her-adjoint}} \int_{\RR} \Psi^* \op F \Psi \dd x = \avgOp{F}.
\]
Так как сопряжение равно себе же, то это действительная величина.
\end{proof}
\end{theorem}

\begin{definition}
\[
\avgOp{F^2} = \int_{\RR} \left|\op F \Psi \right|^2 \dd x 
\]
\begin{proof}
\[
  \avgOp{F^2}
  = \left(\op F \op F \Psi, \Psi\right)
  \by{\eqref{prop:her-adjoint}} \left(\op F \Psi, \op F \Psi\right)
  = \int_{\RR} \left|\op F \Psi\right|^2 \dd x
\]
\end{proof}
\end{definition}

\begin{definition}
\begin{equation}\label{prop:sigma2}
\avgOp{(\Delta F)^2} = \avgOp{\left(F - \avgOp{F}\right)^2} = \avgOp{F^2} - \avgOp{F}^2.
\end{equation}
\begin{proof}
\[
\begin{split}
  & \avgOp{(\Delta F)^2} 
  = \left(\left(F - \avgOp{F}\right)^2 \Psi, \Psi\right)
  = \left(\left(F^2 - 2 F \avgOp{F} + \avgOp{F}^2\right)\Psi, \Psi\right) \\
  & = \left(F^2 \Psi, \Psi\right) - 2 \avgOp{F} \left(F \Psi, \Psi\right) + \avgOp{F}^2 \left(\Psi, \Psi\right)
  = \avgOp{F^2} - 2 \avgOp{F}^2 + \avgOp{F}^2 \cdot 1
  = \avgOp{F^2} - \avgOp{F}^2.
\end{split}
\]
\end{proof}
\end{definition}

\begin{problem}{}
\[
  \Psi(x) = \frac1{\sqrt{x_0 \sqrt\pi}} \exp\left(-\frac{x^2}{2 x_0^2} + i k_0 x\right).
\]

\subproblem Найти \(\avgOp{x}\).
\subproblem Найти \(\avgOp{(\Delta x)^2}\).
\subproblem Найти \(\avgOp{p_x}\).
\subproblem Найти \(\avgOp{(\Delta p_x)^2}\).

\begin{solution}

\subproblem
\[
  \avgOp{x}
  = \int_{\RR} \Psi^* \cdot x \cdot \Psi \dd x
  = \frac1{\sqrt{x_0 \sqrt\pi}} \int_{\RR} x \exp\left(-\frac{x^2}{x_0^2}\right) \dd x = 0.
\]

\subproblem
\[
\avgOp{(\Delta x)^2}
  = \avgOp{x^2} - \avgOp{x}^2 = \avgOp{x^2}
  = \frac{1}{\sqrt{x_0 \sqrt\pi}} \int_{\RR} x^2 \exp\left(-\frac{x^2}{x_0^2}\right) \dd x
  = \frac{x_0^3}{x_0 \sqrt\pi} \frac{\sqrt\pi}{2} = \frac{x_0^2}{2}
\]

\subproblem
\[
\begin{split}
  & \avgOp{p_x}
  = -i \hbar \int_{\RR} \Psi^* \frac{\partial \Psi}{\partial x} \dd x
  \by{\eqreflocal{1}} -\frac{i \hbar}{x_0\sqrt\pi} \int_{\RR} \exp\left(-\frac{x^2}{x_0^2}\right) \left(-\frac{x}{x_0^2} + i k_0\right) \dd x \\
  &\by{\eqreflocal{2}} \frac{\hbar k_0}{\sqrt\pi} \int_{\RR} \exp\left(-\frac{x^2}{x_0^2}\right) \dd \frac{x}{x_0}
  \by{\eqreflocal{3}} \frac{\hbar k_0}{\sqrt\pi} \sqrt\pi
  = \hbar k_0.
\end{split}
\]

\begin{equation}\labellocal{1}
\begin{split}
  & \Psi^* \frac{\partial \Psi}{\partial x}
  = \Psi^* \frac{1}{\sqrt{x_0 \sqrt\pi}} \exp\left(-\frac{x^2}{2x_0^2} + i k_0 x\right) \left(-\frac{x}{x_0^2} + i k_0\right) \\
  &= \frac{1}{\sqrt{x_0 \sqrt\pi}} \exp\left(-\frac{x^2}{2x_0^2} - i k_0 x\right) \cdot \frac{1}{\sqrt{x_0 \sqrt\pi}} \exp\left(-\frac{x^2}{2x_0^2} + i k_0 x\right) \left(-\frac{x}{x_0^2} + i k_0\right) \\
  &= \frac{1}{x_0 \sqrt\pi} \exp\left(-\frac{x^2}{x_0^2}\right) \left(-\frac{x}{x_0^2} + i k_0\right).
\end{split}
\end{equation}

\begin{equation}\labellocal{2}
\begin{split}
  \int_{\RR} \exp\left(-\frac{x^2}{x_0^2}\right) \frac{x}{x_0^2} \dd x
  = \int_{\RR} \mathrm{чёт} \cdot \mathrm{нечёт}\ \dd x
  = \int_{\RR} \mathrm{нечёт}\ \dd x
  = 0.
\end{split}
\end{equation}

\begin{equation}\labellocal{3}
\begin{split}
  \int_{\RR} \exp\left(-\frac{x^2}{x_0^2}\right) \dd \frac{x}{x_0}
  = \left[u = \frac{x}{x_0} \right]
  = \int_{\RR} \exp\left(-u^2\right) \dd u
  \by{\eqref{eq:gauss-int}} \sqrt\pi
\end{split}
\end{equation}

\subproblem
\[\avgOp{\left(\Delta p_x\right)^2} \by{\eqref{prop:sigma2}} \avgOp{p_x^2} - \avgOp{p_x}^2 = \frac{\hbar^2}{2 x_0^2} + \hbar^2 k_0^2 - \left(\hbar k_0\right)^2 = \frac{\hbar^2}{2 x_0^2}.\]

\[
\begin{split}
  & \avgOp{p_x^2}
  = \left(- i \hbar\right)^2 \int_{\RR} \Psi^* \frac{\partial^2 \Psi}{\partial x^2} \dd x
  = \begin{bmatrix}
    u = \Psi^* & \dd u = \frac{\partial \Psi^*}{\partial x} \dd x \\
    v = \frac{\partial \Psi}{\partial x} & \dd v = \frac{\partial^2 \Psi}{\partial x^2} \dd x
  \end{bmatrix}
  = \left(- i \hbar\right)^2 \left(\left.u v\right|_{\RR} - \int_{\RR} v \dd u \right) \\
  &= \left.\left(- i \hbar\right)^2 \Psi^* \frac{\partial \Psi}{\partial x}\right|_{\RR}
  - \left(- i \hbar\right)^2 \int_{\RR} \frac{\partial \Psi}{\partial x} \frac{\partial \Psi^*}{\partial x} \dd x
  \by{\eqreflocal{1}} \hbar^2 \int_{\RR} \frac{\partial \Psi}{\partial x} \frac{\partial \Psi^*}{\partial x} \dd x \\
  &= \frac{\hbar^2}{x_0 \sqrt\pi} \int_{\RR} \exp\left(-\frac{x^2}{x_0^2}\right) \left|-\frac{x}{x_0^2} + i k_0\right|^2 \dd x
  = \frac{\hbar^2}{x_0 \sqrt\pi} \int_{\RR} \exp\left(-\frac{x^2}{x_0^2}\right) \left(\frac{x^2}{x_0^4} + k_0^2\right) \dd x \\
  &= \frac{\hbar^2}{x_0 \sqrt\pi} \int_{\RR} \exp\left(-\frac{x^2}{x_0^2}\right) \left(\frac{x^2}{x_0^4} + k_0^2\right) \dd x
  \by{\eqreflocal{2}, \eqreflocal{3}} \frac{\hbar^2}{x_0 \sqrt\pi} \left(\frac{\sqrt\pi}{2 x_0} + k_0^2 x_0 \sqrt\pi \right) \\
  &= \frac{\hbar^2}{2 x_0^2} + \hbar^2 k_0^2
\end{split}
\]

\begin{equation}\labellocal{1}
  \left.\Psi^* \frac{\partial \Psi}{\partial x}\right|_{\RR}
  = \left.\frac{1}{x_0 \sqrt\pi} \exp\left(-\frac{x^2}{x_0^2}\right) \left(-\frac{x}{x_0^2} + i k_0\right)\right|_{\RR}
  = 0
\end{equation}

\begin{equation}\labellocal{2}
\begin{split}
&\int_{\RR} \exp\left(-\frac{x^2}{x_0^2}\right) \frac{x^2}{x_0^4} \dd x
= \frac{1}{x_0^4} \int_{\RR} \exp\left(-\frac{x^2}{x_0^2}\right) x^2 \dd x \\
&= \begin{bmatrix}
u = x & \dd u = \dd x \\
\dd v = x \exp\left(-\frac{x^2}{x_0^2}\right) \dd x & v = -\frac{x_0^2}{2} \exp\left(-\frac{x^2}{x_0^2}\right)
\end{bmatrix} \\
&= \frac{1}{x_0^4} \left(\left.u v\right|_{\RR} - \int_{\RR} v \dd u\right)
= \frac{1}{x_0^4} \left(\left.-\frac{x x_0^2}{2} \exp\left(-\frac{x^2}{x_0^2}\right)\right|_{\RR} - \int_{\RR} v \dd u\right)
= \frac{1}{x_0^4} \left(0 - \int_{\RR} v \dd u\right) \\
&= \frac{1}{x_0^4} \frac{x_0^2}{2} \int_{\RR} \exp\left(-\frac{x^2}{x_0^2}\right) \dd x
= \frac{1}{x_0^4} \frac{x_0^2}{2} x_0 \int_{\RR} \exp\left(-\frac{x^2}{x_0^2}\right) \dd \frac{x}{x_0}
\by{\eqref{eq:gauss-int}} \frac{1}{x_0^4} \frac{x_0^2}{2} x_0 \sqrt\pi
= \frac{\sqrt\pi}{2 x_0}.
\end{split}
\end{equation}

\begin{equation}\labellocal{3}
\int_{\RR} \exp\left(-\frac{x^2}{x_0^2}\right) k_0^2 \dd x
= x_0 k_0^2 \int_{\RR} \exp\left(-\frac{x^2}{x_0^2}\right) \dd \frac{x}{x_0}
\by{\eqref{eq:gauss-int}} x_0 k_0^2 \sqrt\pi
\end{equation}

\end{solution}
\end{problem}

\begin{definition}
\(\Psi_F\) --- собственная функция и \(F\) --- собственное число:
\[\op F \Psi_F = F \Psi_F.\]
\end{definition}

\begin{problem}{}

\[ \Psi = \exp\left(-\frac{x^2}{2}\right). \]

\[ \op F = \frac{\partial^2}{\partial x^2} - x^2. \]

Найти собственное значение.

\begin{solution}
\[
\begin{split}
  &\op F \Psi
  = \frac{\partial^2}{\partial x^2} \left(\exp\left(-\frac{x^2}{2}\right)\right) - x^2 \exp\left(-\frac{x^2}{2}\right)
  = \left(-x \exp\left(-\frac{x^2}{2}\right)\right)' - x^2 \exp\left(-\frac{x^2}{2}\right) \\
  &= -\exp\left(-\frac{x^2}{2}\right) + x^2 \exp\left(-\frac{x^2}{2}\right) - x^2 \exp\left(-\frac{x^2}{2}\right)
  = -\exp\left(-\frac{x^2}{2}\right).
\end{split}
\]

Таким образом, собственное значение \(F = -1\).
\end{solution}
\end{problem}

\section{Уравнение Шрёдингера}

\begin{equation}
  \specialnumber{Уравнение Шрёдингера}
  i \hbar \frac{\partial \Psi(\vec{r}, t)}{\partial t} = \HHam \Psi
\end{equation}

\begin{equation}
  \specialnumber{Уравнение непрерывности}
  \frac{\partial \omega}{\partial t} + \nabla \cdot \vec{j} = 0,
\end{equation}
где \(\omega = \left|\Psi\right|^{2}\) и \(\bold{j} = \frac{\hbar^{2}}{2 m i} \left[ \Psi^{*} \Laplace \Psi - \Psi \Laplace \Psi^{*}\right]\).
Важно, что \(j \sim \left|A\right|^{2}\).

\begin{remark}
  Поток вероятности определяется как
  \[
    \Pi = \frac{\dd \omega}{\dd t}.
  \]
  Таким образом по \(\Pi\) можно понять тенденцию к изменению плотности вероятности в данной точке.
  Но этого недостаточно, хочется ещё знать, куда именно утекает/откуда приходит плотность вероятности.
  Тогда вводят вектор плотности потока вероятности \(\vec{j}\).
  \[
    \Pi = - \oint_{S} \vec{j} \dd \vec{S}.
  \]
  Минус здесь нужен по следующим соображениям: если плотность вероятности утекает из точки, то \(\Pi\) отрицательно,
  однако \(\vec{j}\) будет направлен от точки, следовательно, интеграл даст положительное значение.
  Тогда можно применить теорему Остроградского--Гаусса:
  \[
    \Pi = \frac{\dd \omega}{\dd t} = - \oint_{S} \vec{j} \dd \vec{S} = - \int_{V} \nabla \cdot \vec{j} \dd V
  \]
  Следовательно,
  \[
    \int_{V} \left(\frac{\dd \omega}{\dd t} + \nabla \cdot \vec{j}\right) \dd V = 0.
  \]
  А значит в любой точке:
  \[
    \frac{\dd \omega}{\dd t} + \nabla \cdot \vec{j} = 0
  \]
  Ещё плотность потока определяет, сколько частиц пройдёт в единицу времени через поперечное сечение единичной площади.
\end{remark}

\begin{equation}\label{eq:stationary-sch-eq}
  \specialnumber{Стационарное уравнение Шрёдингера}
  -\frac{\hbar^{2}}{2m}\Laplace \psi(\vec{r}) + U(\vec{r})\psi(\vec{r}) = E \psi(\vec{r}).
\end{equation}
Как видно, \(U\) не зависит от времени.
Идея в том, чтобы представить волновую функцию как \(\Psi(\vec{r}, t) = \psi(\vec{r}) \exp\left(-i \frac{E}{\hbar} t\right)\).
Тогда у волновой функции будут приятные свойства: \(\omega\) и среднее значение не зависят от времени.
Однако если мы из волновых функций вида \(\Psi(\vec{r}, t) = \psi(\vec{r}) \exp\left(-i \frac{E}{\hbar} t\right)\)
составим линейную комбинацию, то этих приятных свойств может и не быть.

\begin{remark}
  Чтобы получить стационарное уравнение Шрёдингера, нужно прибегнуть к методу разделения переменных:
  \[
    \Psi(x, t) = \psi(x) \phi(t).
  \]
  Тогда уравнение Шрёдингера после упрощение будет иметь вид:
  \[
    i \hbar \psi \frac{\dd \phi}{\dd t} = -\frac{\hbar^{2} \phi}{2m} \frac{\dd^{2} \psi}{\dd x^{2}} + U \psi \phi
  \]
  Поделим на \(\psi \phi\), чтобы левая часть зависела только от \(t\), а правая только от \(x\).
  \[
    i \hbar \frac{1}{\phi} \frac{\dd phi}{\dd t} = -\frac{\hbar^{2}}{2m \psi} \frac{\dd^{2} \psi}{\dd x^{2}} + U
  \]
  Но так как это функции от разных переменных, то обе части должны быть равны какой-то константе. Обозначим её \(E\).
  В итоге получим два уравнения:
  \[
    \begin{cases}
      i \hbar \frac{1}{\phi} \frac{\dd \phi}{\dd t} = E, \\
      -\frac{\hbar^{2}}{2m \psi} \frac{\dd^{2} \psi}{\dd x^{2}} + U = E.
    \end{cases}
  \]
  Решением первого уравнение будет: \(\phi(t) = \exp\left(-i \frac{E}{\hbar} t\right)\).
  Решение второго уравнения будет зависеть от формы \(U\).
\end{remark}


\begin{problem}{Свободное одномерное движение частицы}
Частица массы \(m\) движется в свободно (\(U = 0\)). Найти \(E\) и \(\Psi\) в одномерном случае.

\begin{solution}
  Воспользуемся уравнением~\eqref{eq:stationary-sch-eq}:
  \[
    -\frac{\hbar^{2}}{2m} \Laplace \psi(\vec{r}) = E \psi(\vec{r}).
  \]
  \[
    -\frac{\hbar^{2}}{2m} \frac{\dd^{2} \psi}{\dd x^{2}} = E \psi(x).
  \]
  Решением данного дифференциального уравнения будет:
  \[
    \psi^{(\pm)}(x) = A^{(\pm)} \exp\left(\pm i \frac{\sqrt{2 m E}}{\hbar} x\right)
  \]
\end{solution}
\end{problem}

\begin{remark}
  Вообще говоря решением будет:
  \[
    \psi(x) = A e^{i k x} + B e^{-i k x},
  \]
  где \(k = \frac{\sqrt{2 m E}}{\hbar}\).
  Однако можно принять, что \(k = \pm \frac{\sqrt{2 m E}}{\hbar}\) и тогда
  \[
    \psi(x) = A e^{i k x}.
  \]
  Тогда волновая функция будет иметь вид:
  \begin{equation}\label{eq:free-particle-psi-k}
    \Psi_{k}(x, t) = A \exp\left(i \left(k x - \frac{\hbar k^{2}}{2m} t\right)\right).
  \end{equation}
  Энергия же может принимать любые значения, на неё нет никаких ограничений.
  Найдём \(A\), воспользуемся \ref{prop:psi-1}:
  \[
    |A|^{2} \int_{\RR} e^{2 i k x - i \frac{E}{\hbar} t}  e^{-2 i k x + i \frac{E}{\hbar} t} \dd x = |A|^{2} \int_{\RR} \dd x = |A|^{2} \cdot (\infty).
  \]
  Таким образом, волновая функция~\eqref{eq:free-particle-psi-k} не может (физически) представлять состояние системы, так как она не нормируема.
  Однако
  \begin{align*}
    \Psi(x, t) &= \frac{1}{\sqrt{2 \pi}} \int_{\RR} \phi(k) e^{i (k x + \frac{\hbar k^{2}}{2m} t)} \dd k, \\
    \phi(k) &= \frac{1}{\sqrt{2 \pi}} \int_{\RR} \Psi(x, 0) e^{-i k x}\dd x.
  \end{align*}
\end{remark}

\begin{problem}{Потенциальный барьер}
  Пусть задана функция:
  \[
    U(x) =
    \begin{cases}
      0, & x < 0, \\
      U_{0}, & 0 \leq x \leq a, \\
      0, & x > a.
    \end{cases}
  \]
  Пусть \(0 < E < U_{0}\). Найти волновую функцию.

\begin{solution}
  Найдём волновую функцию, решив стационарное уравнение Шрёдингера. Разобъём область на три части.
  \begin{align*}
    & -\frac{\hbar^{2}}{2m} \frac{\dd^{2} \psi_{\RNum{1}, \RNum{3}}}{\dd x^{2}} = E \psi_{\RNum{1}, \RNum{3}} \\
    & -\frac{\hbar^{2}}{2m} \frac{\dd^{2} \psi_{\RNum{2}}}{\dd x^{2}} + U_{0} \psi_{\RNum{2}} = E \psi_{\RNum{2}}.
  \end{align*}
  Решениями будут:
  \begin{align*}
    & \psi_{\RNum{1}} = A_{1} e^{i k x} + B_{1} e^{- i k x}, & k = \frac{\sqrt{2 m E}}{\hbar}, \\
    & \psi_{\RNum{2}} = A_{2} e^{-k_{2} x} + B_{2} e^{k_{2} x}, & k_{2} = \frac{\sqrt{2 m (U_{0} - E)}}{\hbar}, \\
    & \psi_{\RNum{3}} = A_{3} e^{i k (x - a)}, & k = \frac{\sqrt{2 m E}}{\hbar}.
  \end{align*}
  В третьем уравнении нет второго слагаемого, так как предполагается, что волна идёт вправо.
  Из условия непрерывности волновой функции:
  \begin{align*}
    A_{1} + B_{1} &= A_{2} + B_{2}, \\
    A_{2} e^{-k_{2} a} + B_{2} e^{k_{2} a} &= A_{3}, \\
    i k (A_{1} - B_{1}) &= k_{2} (B_{2} - A_{2}), \\
    k_{2} (B_{2} e^{k_{2} a} - A_{2} e^{-k_{2} a}) &= i k A_{3}.
  \end{align*}
  Отсюда:
  \begin{align*}
    A_{2} &= \frac{1 - i n}{2} e^{k_{2} a} A_{3}, \\
    B_{2} &= \frac{1 + i n}{2} e^{-k_{2} a} A_{3}.
  \end{align*}
  Систему из четырёх уравнений можно: если \(k_{2} a > 3\), то \(\left|A_{2}\right| >> \left|B_{2}\right|\); таким образом пренебрегаем \(B_{2}\).

  В таком случае:
  \begin{align*}
    A_{1} &= \frac{(1 - i n) (n + i)}{2 n} e^{k_{2} a} A_{3}, \\
    B_{1} &= \frac{(1 - i n) (n - i)}{2 n} e^{k_{2} a} A_{3}.
  \end{align*}
  \begin{align*}
    R &= \left|\vec{j_{\text{отр}}}\right| / \left|\vec{j_{0}}\right| & \quad \text{Коэффициент отражения}, \\
    D &= \left|\vec{j_{\text{пр}}}\right| / \left|\vec{j_{0}}\right| & \quad \text{Коэффициент прохождения}.
  \end{align*}
  \(\vec{j_{\text{отр}}}\) --- вектор плотности потока частиц, отражённых от барьера,
  \(\vec{j_{0}}\) --- вектор плотности потока частиц, движущихся к барьеру,
  \(\vec{j_{\text{пр}}}\) --- вектор плотности потока частиц, прошедших барьер.

  Так как в данном случае \(\vec{j}\) пропорционален квадрату амплитуды, то
  \begin{align*}
    R &= \frac{|B_{1}|^{2}}{|A_{1}|^{2}} \\
    D &= \frac{|A_{3}|^{2}}{|A_{1}|^{2}} = \frac{16 n^{2}}{{(1 + n^{2})}^{2}} e^{-2 k_{2} a}.
  \end{align*}
  где \(n = \frac{k}{k_{2}} = \sqrt{\frac{E}{U_{0} - E}}\).

  Например, если \(U_{0} - E = \qty{1}{\eV}\) и \(a = \qty{0.3}{\nano\metre}\), то \(D \approx \num{0.05}\).
\end{solution}
\end{problem}
\end{document}
